\documentclass[letter]{book}

\usepackage{fullpage} % Package to use full page
\usepackage{parskip} % Package to tweak paragraph skipping
\usepackage{tikz} % Package for drawing

\usepackage{amssymb}
\usepackage{amsmath}
\usepackage{amscd}

%https://en.wikibooks.org/wiki/LaTeX/Theorems
\usepackage{amsthm}
\theoremstyle{definition}
\newtheorem{theorem}{Theorem}[chapter]
\newtheorem{lemma}[theorem]{Lemma}
\newtheorem{corollary}[theorem]{Corollary}

\theoremstyle{definition}
\newtheorem{mydef}{Definition}[chapter]
\newtheorem*{ex}{Example}

\newtheorem{prop}{Proposition}[chapter]
\theoremstyle{remark}
\newtheorem*{rem}{Remark}

\usepackage{enumitem}

\usepackage{tkz-graph}
\usepackage{float}

\usepackage{hyperref}

\usepackage{MnSymbol}

%https://tex.stackexchange.com/questions/229355/algorithm-algorithmic-algorithmicx-algorithm2e-algpseudocode-confused
\usepackage{algorithm}
\usepackage{algorithmicx}
\usepackage{algpseudocode}

\usepackage{graphicx}
\graphicspath{ {./resources/} }

%https://tex.stackexchange.com/questions/165021/fixing-the-location-of-the-appearance-in-algorithmicx-environment
\usepackage{float}% http://ctan.org/pkg/float

%https://tex.stackexchange.com/questions/25369/how-to-rotate-a-table
\usepackage[graphicx]{realboxes}

\begin{document}
\setcounter{chapter}{33}

\chapter{NP-Completeness}
\section*{$2$-CNF satisfiability vs. $3$-CNF satisfiability}
$k$-CNF: A boolean formula is in \textit{\textbf{$k$-conjunctive normal form}} if it is the AND of clauses of ORs of exactly $k$ variables or their negations.

\section*{NP-completeness and the class P and NP}
\begin{enumerate}
    \item Optimization problems $\Rightarrow$ decision problems
    \item Reductions
        \begin{enumerate}
            \item The transformation takes polynomial time.
            \item The answer are the same. That is, the answer for $\alpha$ is ``yes" if and only if the answer for $\beta$ is also ``yes."
        \end{enumerate}
\end{enumerate}
\section*{Reduction algorithm}
\begin{enumerate}
    \item Given an instance $\alpha$ of problem $A$, use a polynomial-time reduction algorithm to transform it to an instance $\beta$ of problem $B$.
    \item Run the polynomial-time decision algorithm for $B$ on the instance $\beta$.
    \item Use the answer for $\beta$ as the answer for $\alpha$.
\end{enumerate}
\section{Polynomial time}
\section{Polynomial-time verification}
\section{NP-completeness and reducibility}
\subsection{Reducibility}
A language $L_1$ is polynomial-time reducible to a language $L_2$, written 
    \begin{equation*}
        L_1\leq_p L_2       
    \end{equation*}
if there exists a polynomial-time computable function $f:\{0,1\}^*\rightarrow\{0,1\}^*$ such that for all $x\in \{0,1\}^*, x\in L_1$ if and only if $f(x)\in L_2$.

\section{NP-completeness proofs}
\begin{enumerate}
    \item Prove $L\in NP$.
    \item Select a known NP-complete language $L'$.
    \item Describe an algorithm that computes a function $f$ mapping every instance $x\in\{0,1\}^*$ of $L'$ to an instance $f(x)$ of $L$.
    \item Prove that the function $f$ satisfies $x\in L'$ if and only if $f(x)\in L$ for all $x\in \{0,1\}^*$.
    \item Prove that the algorithm computing $f$ runs in a polynomial time.
\end{enumerate}

\section{NP-complete problems}
\includegraphics[scale=0.6]{"Figure34-13"}
\subsection{The clique problem}
A \textit{clique} in an undirected graph $G=(V,E)$ is a subset $V'\subseteq V$ of vertices, each pair of which is connected by an edge in $E$; that is, a clique is a complete subgraph of $G$. The \textit{clique problem} is the optimization problem of finding a clique of maximum size in a graph.\\
CLIQUE$=\{\langle G, K\rangle:G$ is a graph containing a clique of size $k\}$.
\bigskip
\setcounter{theorem}{10}
\begin{theorem}
    The clique problem is NP-complete.
\end{theorem}
\begin{proof} Two steps.\\
    \begin{enumerate}
        \item Show that CLIQUE $\in$ NP, for a given graph $G=(V,E)$: we can check whether a solution $V'$ is a clique in polynomial time by checking whether, for each pair of $u,v\in V'$, the edge $(u,v)\in E$.
        \item Prove that $3$-CNF-SAT $\leq_p$ CLIQUE.\\
        Let $\phi=C_1\land C_2\land\cdots\land C_k$ be a boolean formula in $3$-CNF with $k$ clauses. For $r=1,2,\ldots,k$, each clause $C_r$ has exactly three distinct literals $l_1^r,l_2^r$ and $l_3^r$. We shall construct a graph $G$ such that $\phi$ is satisfiable if and only if $G$ has a clique of size $k$.
    \end{enumerate}
\end{proof}
\subsection{The vertex-cover problem}
A vertex cover of an undirected graph $G=(V,E)$ is subset $V'\subseteq V$ such that if $(u,v)\in E$, then $u\in V'$ or $v\in V'$ (or both).
\bigskip
\begin{theorem}
    The vertex-cover problem is NP-complete.
\end{theorem}
\begin{proof}
    
\end{proof}
\subsection{The hamiltonian-cycle problem}
VERTEX-COVER $\leq_p$ HAM-CYCLE
\subsection{The traveling-salesman problem}
HAM-CYCLE $\leq_p$ TSP
\subsection{The subset-sum problem}
SUBSET-SUM $=\{\langle S,t\rangle:$ there exists a subset $S'\subseteq S$ such that $t=\sum_{s\in S'}{s}\}$.
\setcounter{theorem}{14}
\begin{theorem}
    The subset-sum problem is NP-complete.
\end{theorem}
\begin{proof} Two steps.\\
    \begin{enumerate}
        \item To show that SUBSET-SUM is in NP, for an instance $\langle S,t\rangle$ of the problem, we let the subset $S'$ be the certificate. A verification algorithm can check whether $t=\sum_{s\in S'}{s}$ in polynomial time.
        \item Show that $3$-CNF-SAT $\leq_p$ SUBSET-SUM.\\
            Given a $3$-CNF formula $\phi$ over variables $x_1,x_2,\ldots,x_n$ with clauses $C_1,C_2,\ldots, C_k$, each containing exactly three distinct literals, the reduction algorithm constructs an instance $\langle S,t\rangle$ of the subset-sum problem such that $\phi$ is satisfiable if and only if there exists a subset of $S$ whose sum is exactly $t$.
            \begin{enumerate}
                \item Without loss of generality, we make two simplifying assumptions about the formula $\phi$.
                    \begin{enumerate}
                        \item No clause contains both $x_i$ and $\Bar{x_i}$.
                        \item For all $x_i$, it appears in at least one clause.
                    \end{enumerate}
                \item The reduction create two numbers in set $S$ for each variable $x_i$ and two numbers in $S$ for each clause $C_j$, each number contains $n+k$ digits. We construct set $S$ and target $t$ as follows.
                    \begin{enumerate}
                        \item The target $t$ has a $1$ in each digit labeled by a variable and a $4$ in each digit labeled by a clause.
                        \item For each variable $x_i$, set $S$ contains two integer $v_i$ and $v_i'$. Each of $v_i$ and $v_i'$ has a $1$ in the digit labeled by $x_i$ and $0$s in the other variable digits.
                        \item If literal $x_i$ appears in clause $C_j$, then the digit labeled by $C_j$ in $v_j$ contains a $1$. If literal $\lnot{x_i}$ appears in clause $C_j$, then the digit labeled by $C_j$ in $v_i'$ contains a $1$. All other digits labeled by clauses in $v_i$ and $v_i'$ are $0$.
                        \item For each clause $C_j$, set $S$ contains two integers $s_j$ and $s_j'$. Each of $s_j$ and $s_j'$ has $0$s in all digits other than the one labeled by $C_j$. For $s_j$, there is a $1$ in the $C_j$ digit, and $s_j'$ has a $2$ in this digit.
                    \end{enumerate}
                    \begin{table}[h]
                        \centering
                        \begin{tabular}{ccccccccc}
                              &   & $x_1$ & $x_2$ & $x_3$ & $C_1$ & $C_2$ & $C_3$ & $C_4$\\
                            \hline\hline
                            $v_1$ & = & 1 & 0 & 0 & 1 & 0 & 0 & 1\\
                            $v_1'$ & = & 1 & 0 & 0 & 0 & 1 & 1 & 0\\
                            \hline
                            $v_2$ & = & 0 & 1 & 0 & 0 & 0 & 0 & 1\\
                            $v_2'$ & = & 0 & 1 & 0 & 1 & 1 & 1 & 0\\
                            \hline
                            $v_3$ & = & 0 & 0 & 1 & 0 & 0 & 1 & 1\\
                            $v_3'$ & = & 0 & 0 & 1 & 1 & 1 & 0 & 0\\
                            \hline
                            $s_1$ & = & 0 & 0 & 0 & 1 & 0 & 0 & 0\\
                            $s_1'$ & = & 0 & 0 & 0 & 2 & 0 & 0 & 0\\
                            \hline
                            $s_2$ & = & 0 & 0 & 0 & 0 & 1 & 0 & 0\\
                            $s_2'$ & = & 0 & 0 & 0 & 0 & 2 & 0 & 0\\
                            \hline
                            $s_3$ & = & 0 & 0 & 0 & 0 & 0 & 1 & 0\\
                            $s_3'$ & = & 0 & 0 & 0 & 0 & 0 & 2 & 0\\
                            \hline
                            $s_4$ & = & 0 & 0 & 0 & 0 & 0 & 0 & 1\\
                            $s_4'$ & = & 0 & 0 & 0 & 0 & 0 & 0 & 2\\
                            \hline\hline
                            $t$ & = & 1 & 1 & 1 & 4 & 4 & 4 & 4
                        \end{tabular}
                        \caption{$\phi=(x_1\lor\lnot{x_2}\lor\lnot{x_3})\land(\lnot{x_1}\lor\lnot{x_2}\lor\lnot{x_3})\land(\lnot{x_1}\lor\lnot{x_2}\lor{x_3})\land({x_1}\lor{x_2}\lor{x_3})$. A satisfying assignment of $\phi$ is $\langle x_1=0,x_2=1,x_3=1\rangle$}
                        \label{tab:my_label}
                    \end{table}
            \end{enumerate}
    \end{enumerate}
\end{proof}
\end{document}