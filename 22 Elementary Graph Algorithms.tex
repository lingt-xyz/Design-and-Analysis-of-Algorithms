\documentclass[letter]{book}

\usepackage{fullpage} % Package to use full page
\usepackage{parskip} % Package to tweak paragraph skipping
\usepackage{tikz} % Package for drawing

\usepackage{amssymb}
\usepackage{amsmath}
\usepackage{amscd}

%https://en.wikibooks.org/wiki/LaTeX/Theorems
\usepackage{amsthm}
\theoremstyle{definition}
\newtheorem{theorem}{Theorem}[chapter]
\newtheorem{lemma}[theorem]{Lemma}
\newtheorem{corollary}[theorem]{Corollary}

\theoremstyle{definition}
\newtheorem{mydef}{Definition}[chapter]
\newtheorem*{ex}{Example}

\newtheorem{prop}{Proposition}[chapter]
\theoremstyle{remark}
\newtheorem*{rem}{Remark}

\usepackage{enumitem}

\usepackage{tkz-graph}
\usepackage{float}

\usepackage{hyperref}

\usepackage{MnSymbol}

%https://tex.stackexchange.com/questions/229355/algorithm-algorithmic-algorithmicx-algorithm2e-algpseudocode-confused
\usepackage{algorithm}
\usepackage{algorithmicx}
\usepackage{algpseudocode}

\usepackage{graphicx}
\graphicspath{ {./resources/} }

%https://tex.stackexchange.com/questions/165021/fixing-the-location-of-the-appearance-in-algorithmicx-environment
\usepackage{float}% http://ctan.org/pkg/float

%https://tex.stackexchange.com/questions/25369/how-to-rotate-a-table
\usepackage[graphicx]{realboxes}

\begin{document}
\setcounter{chapter}{21}

\chapter{Elementary Graph Algorithms}

\section{Representations of graphs}
\subsection{Adjacency list representation}
\subsection{Adjacency matrix representation}

\section{Breadth-first search}
$O(V+E)$ in the size of the adjacency-list representation of $G$.
\bigskip
\begin{lemma}

\end{lemma}
\bigskip
\begin{lemma}

\end{lemma}
\bigskip
\begin{lemma}

\end{lemma}
\bigskip
\begin{corollary}

\end{corollary}
\bigskip
\begin{theorem}

\end{theorem}
\subsection{Breadth-first trees}
\begin{lemma}

\end{lemma}


\section{Depth-first search}
    \begin{theorem} (Parenthesis theorem)\\
        In any depth-first search of a (directed or undirected) graph $G=(V,E)$, for any two vertices $u$ and $v$, exactly one of the following three conditions holds:
        \begin{itemize}
            \item the interval $[u.d, u.f]$ and $[v.d, v.f]$ are entirely \textcolor{red}{disjoint}, and neither $u$ or $v$ is a descendant of the other in the depth-first forest,
            \item the interval $[u.d, u.f]$ is \textcolor{red}{contained} entirely within the interval $[v.d, v.f]$, and $u$ is a descendant of $v$ in a depth-first tree, or
            \item the interval $[v.d, v.f]$ is \textcolor{red}{contained} entirely within the interval $[u.d, u.f]$, and $v$ is a descendant of $u$ in a depth-first tree.
        \end{itemize}
    \end{theorem}
    \bigskip
    \begin{corollary} %(Nesting of descendants' intervals)
    
    \end{corollary}
    \bigskip
    \begin{theorem}
        
    \end{theorem}
    \bigskip
    \begin{theorem}
        
    \end{theorem}

\section{Topological sort}
A topological sort of a dag $G=(V,E)$ is a linear ordering of all its vertices such that if $G$ contains an edge $(u,v)$, then $u$ appears before $v$ in the ordering.
\bigskip
\begin{lemma}

\end{lemma}
\bigskip
\begin{theorem}
    
\end{theorem}

\section{Strongly connected components}
A strongly connected component of a directed graph $G=(V,E)$ is a maximal set of vertices $C\subseteq V$ such that for every pair of vertices $u$ and $v$ in $C$, we have both $u\rightlsquigarrow v$ and $v\rightlsquigarrow u$; that is, vertices $u$ and $v$ are reachable from each other.
\bigskip
\begin{lemma}

\end{lemma}
\bigskip
\begin{lemma}

\end{lemma}
\bigskip
\begin{corollary}

\end{corollary}
\bigskip
\begin{theorem}
    
\end{theorem}
\end{document}