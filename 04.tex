\documentclass[a4paper]{article}

\usepackage{fullpage} % Package to use full page
\usepackage{parskip} % Package to tweak paragraph skipping
\usepackage{tikz} % Package for drawing
\usepackage{amsmath}
\usepackage{hyperref}
\usepackage{amssymb}
\usepackage{tikz-qtree}

%https://tex.stackexchange.com/questions/229355/algorithm-algorithmic-algorithmicx-algorithm2e-algpseudocode-confused
\usepackage{algorithm}
\usepackage{algorithmicx}
\usepackage{algpseudocode}

\usepackage{graphicx}
\graphicspath{ {./resources/} }

%https://tex.stackexchange.com/questions/165021/fixing-the-location-of-the-appearance-in-algorithmicx-environment
\usepackage{float}% http://ctan.org/pkg/float

%https://tex.stackexchange.com/questions/25369/how-to-rotate-a-table
\usepackage[graphicx]{realboxes}
\title{4 Divide-and-Conquer}
\author{Ling Tan}
\date{2018-9}

\begin{document}
\maketitle
A  \textit{recurrence} is an equation or inequality that describes a function in terms of its value on smaller inputs.
\section*{4.2 Strassen’s algorithm for matrix multiplication}
\section*{4.3 The substitution method for solving recurrences}
\begin{enumerate}
    \item Guess the form of the solution.
    \item Use mathematical induction to find the constants and show that the solution works.
\end{enumerate}
\subsection*{Example}
$T(n)=2T(\lfloor n/2\rfloor)+n$\\
\begin{enumerate}
    \item Guess $T(n)\in O(n \log n)$
    \item Prove: Assume $T(\lfloor{k}\rfloor)\leq c\lfloor{k}\rfloor\log{\lfloor{k}\rfloor}, k< n\text{, that is, }T(\lfloor{n/2}\rfloor)\leq c\lfloor{n/2}\rfloor\log{\lfloor{n/2}\rfloor}$
    \begin{align*}
 T(n) & = 2T(n/2) + n \\
 & \leq 2(c\lfloor{n/2}\rfloor\log{\lfloor{n/2}\rfloor})+n \\ 
 & \leq cn\log{\lfloor{n/2}\rfloor}+n \\
 & = cn \log{n} -cn\log{2}+2\\
 & = cn \log n - cn+n \\
 & \leq cn \log n \text{ for } c\geq 1
 \end{align*}
 \subsection*{Avoiding Pitfalls}
 Make $n_0$ big enough to avoid boundary problem.
\end{enumerate}
\section*{4.4 The recursion-tree method for solving recurrences}
\section*{4.5 The master method for solving recurrences}
\end{document}