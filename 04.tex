\documentclass[a4paper]{article}

\usepackage{fullpage} % Package to use full page
\usepackage{parskip} % Package to tweak paragraph skipping
\usepackage{tikz} % Package for drawing
\usepackage{amsmath}
\usepackage{hyperref}
\usepackage{amssymb}
\usepackage{tikz-qtree}

%https://tex.stackexchange.com/questions/229355/algorithm-algorithmic-algorithmicx-algorithm2e-algpseudocode-confused
\usepackage{algorithm}
\usepackage{algorithmicx}
\usepackage{algpseudocode}

\usepackage{graphicx}
\graphicspath{ {./resources/} }
\usepackage{amsthm}
\theoremstyle{plain}
\newtheorem*{theorem*}{Theorem}
\newtheorem{theorem}{Theorem}

%https://tex.stackexchange.com/questions/165021/fixing-the-location-of-the-appearance-in-algorithmicx-environment
\usepackage{float}% http://ctan.org/pkg/float

%https://tex.stackexchange.com/questions/25369/how-to-rotate-a-table
\usepackage[graphicx]{realboxes}
\title{4 Divide-and-Conquer}
\author{Ling Tan}
\date{2018-9}

\begin{document}
\maketitle
A  \textit{recurrence} is an equation or inequality that describes a function in terms of its value on smaller inputs.\\
\\
Three methods for solving recurrences:
\begin{enumerate}
    \item Substitution method
    \item Recursion-tree method
    \item Master method
\end{enumerate}
\section*{4.2 Strassen’s algorithm for matrix multiplication}
If $A=(a_{ij}), B=(b_{ij})$ are square $n\times n$ matrices, then
\begin{equation*}
    C=A\times B, \text{ entry }c_{ij}=\sum_{k=1}^n a_{ik}\cdot b_{kj}
\end{equation*}
\begin{itemize}
    \item Square-Matrix-Multiply: Running time $T(n)=\Theta (n^3)$
    \item A simple divide-and-conquer algorithm\\
    Partition each of $A, B, and C$ into four $n/2 \times n/2$ matrices,
    \begin{equation*}
        \begin{pmatrix}
      a & b \\
      c & d \\ 
  \end{pmatrix}
  ,
  \begin{pmatrix}
      e & g \\
      f & h \\ 
  \end{pmatrix}
  ,
  \begin{pmatrix}
      r & s \\
      t & u \\ 
  \end{pmatrix}
    \end{equation*}
    so that we rewrite the equation $C=A\cdot B$ as
    \begin{equation*}
        \begin{pmatrix}
      r & s \\
      t & u \\  
  \end{pmatrix}
  =
  \begin{pmatrix}
      a & b \\
      c & d \\ 
  \end{pmatrix}
  \cdot
  \begin{pmatrix}
      e & g \\
      f & h \\ 
  \end{pmatrix}
    \end{equation*}
    with
    \begin{align*}
        r&=a\cdot e + b\cdot f,\\
  s&=a\cdot g + b\cdot h,\\
  t&=c\cdot e + d\cdot f,\\
  u&=c\cdot g + d\cdot h.
    \end{align*}
    In total, 8 multiplication and 4 adds.
    \begin{align*}
        T(n) & = 8T(n/2) + \Theta(n^2) \\
  & = 8\bigl(8T(n/{2^2}\bigr) + \Theta\bigl((n/2)^2)\bigr) + \Theta(n^2) \\ 
  & = 8^2T(n/{2^2}) + \Theta\bigl((n/2)^2\bigr) + \Theta(n^2) \\
  & = TODO
    \end{align*}
    \item Stassen’s method
    \begin{align*}
  P_1 &= a \cdot (g-h)\\
  P_2 &= (a+b) \cdot h\\
  P_3 &= (c+d) \cdot e\\
  P_4 &= d \cdot (f-e)\\
  P_5 &= (a+d)\cdot (e+h)\\
  P_6 &= (b-d) \cdot (f+h)\\
  P_7 &= (a-e) \cdot (e+g)\\
  &\\
  r&=P_5+P_4-P_2+P_6\\
  s&=P_1+P_2\\
  t&=P_3+P_4\\
  u&=P_5+P_1-P_3-P_7
  \end{align*}
  In total, 7 multiplications, and 18 adds.
  \begin{align*}
  T(n) & = 7T(n/2) + \Theta(n^2) \\
  & = \Theta(n^{\log_2{7}}) \\ 
  & = \Theta(n^{2.81})
  \end{align*}
\end{itemize}
\section*{4.3 The substitution method for solving recurrences}
\begin{enumerate}
    \item Guess the form of the solution.
    \item Use mathematical induction to find the constants and show that the solution works.
\end{enumerate}
\subsection*{Example} 
$T(n)=2T(\lfloor n/2\rfloor)+n$
\begin{enumerate}
    \item Guess $T(n)\in O(n \log n)$
    \item Prove: Assume $T(\lfloor{k}\rfloor)\leq c\lfloor{k}\rfloor\log{\lfloor{k}\rfloor}, k< n\text{, that is, }T(\lfloor{n/2}\rfloor)\leq c\lfloor{n/2}\rfloor\log{\lfloor{n/2}\rfloor}$
    \begin{align*}
 T(n) & = 2T(n/2) + n \\
 & \leq 2(c\lfloor{n/2}\rfloor\log{\lfloor{n/2}\rfloor})+n \\ 
 & \leq cn\log{\lfloor{n/2}\rfloor}+n \\
 & = cn \log{n} -cn\log{2}+2\\
 & = cn \log n - cn+n \\
 & \leq cn \log n \text{ for } c\geq 1
 \end{align*}
 \subsection*{Avoiding Pitfalls}
 Make $n_0$ big enough to avoid boundary problem.
\end{enumerate}
\section*{4.4 The recursion-tree method for solving recurrences}
\includegraphics[scale=0.7]{"recursion-tree"}
\section*{4.5 The master method for solving recurrences}
\includegraphics[scale=0.4]{"Master-Theorem"}
\begin{theorem*}
Let
\begin{equation*}
    T(n) = aT(n/b) + f(n)
\end{equation*}
where $a\geq 1,b\geq 1$.
\begin{enumerate}
    \item If $f(n)/{n^{\log_b{a}}}\in O(n^{-\epsilon})$ for some constant $\epsilon>0$, then $T(n)=\Theta(n^{\log_b{a}})$.
    \item If $f(n)/{n^{\log_b{a}}}\in \Theta({\log^k{n}})$ for some constant $k\geq 0$, then $T(n)=\Theta(n^{\log_b{a}} {\log^{k+1}{n}})$.
    \item If $f(n)/{n^{\log_b{a}}}\in \Omega(n^{\epsilon})$ for some constant $\epsilon>0$, and if $f(n/b)\leq cf(n)$ for some constant $c>0$, then $T(n)=\Theta\bigl(f(n)\bigr)$.
    \item Some are not  in these cases, see example 4.
\end{enumerate}
\end{theorem*}
\subsection*{Examples}
\begin{enumerate}
    \item Master theorem case 1: $T(n) = 2T(n/2) + 1$
    \begin{align*}
        a=2, b=2, f(n)=1,n^{\log_b{a}}=n^{\log_2{2}}=n\\
 f(n)/{n^{\log_b{a}}}=1/n=n^{-1}\in O(n^{-1}), \epsilon=1\\
 \Rightarrow T(n)=\Theta(n^{\log_b{a}})=\Theta(n^1)=\Theta(n)
    \end{align*}
    \item Master theorem case 2: $T(n) = 2T(n/2) + n\log{n}$
    \begin{align*}
        a=2, b=2, f(n)=n\log{n},n^{\log_b{a}}=n^{\log_2{2}}=n\\
 f(n)/{n^{\log_b{a}}}={n\log{n}}/n=\log{n}\in \Theta({\log^1{n}}), k=1\\
 \Rightarrow T(n)=T(n)=\Theta(n^{\log_b{a}} {\log^{k+1}{n}})=\Theta(n {\log^{2}{n}})
    \end{align*}
    \item Master theorem case 3: $T(n) = 3T(n/4) + n\log{n}$
    \begin{gather*}
        a=3, b=4, f(n)=n\log{n},n^{\log_b{a}}=n^{\log_4{3}}=O(n^{0.793})\\
        f(n)/{n^{\log_b{a}}}={n\log{n}}/{n^{0.793}}=\Omega(n^{\epsilon}),\epsilon>0\\
        f(n/b)=(n/b)(\log{n/b})=(n/4)(\log{n/4})\leq (n/4)(\log{n})\leq cf(n)\\
        \Rightarrow T(n)=\Theta(f(n))=\Theta(n\log{n})
    \end{gather*}
    \item Not in these cases: $T(n) = 4T(n/2) + n^2/{\log n}$
    \begin{gather*}
        a=4, b=2, f(n)=n^2/{\log n},n^{\log_b{a}}=n^{\log_2{4}}=n^2\\
 f(n)/{n^{\log_b{a}}}=(n^2/{\log n})/{n^2}=1/\log{n}\\
 \Rightarrow T(n)=???
    \end{gather*}
    \item Master theorem case 1: Strassen $T(n) = 7T(n/2) + \Theta(n^2)$
    \begin{gather*}
        a=7, b=2, f(n)=\Theta(n^2),n^{\log_b{a}}=n^{\log_2{7}}=n^{2.81}\\
        f(n)/{n^{\log_b{a}}}={\Theta(n^2)}/{n^{2.81}}=n^{-k}, k>0\\
        \Rightarrow T(n)=\Theta(n^{\log_b{a}})=\Theta(n^{\log_2{7}})
    \end{gather*}
\end{enumerate}

\end{document}