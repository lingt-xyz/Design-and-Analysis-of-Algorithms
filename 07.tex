\documentclass[a4paper]{article}

\usepackage{fullpage} % Package to use full page
\usepackage{parskip} % Package to tweak paragraph skipping
\usepackage{tikz} % Package for drawing
\usepackage{amsmath}
\usepackage{hyperref}
\usepackage{amssymb}
\usepackage{tikz-qtree}

%https://tex.stackexchange.com/questions/229355/algorithm-algorithmic-algorithmicx-algorithm2e-algpseudocode-confused
\usepackage{algorithm}
\usepackage{algorithmicx}
\usepackage{algpseudocode}

\usepackage{graphicx}
\graphicspath{ {./resources/} }

%https://tex.stackexchange.com/questions/165021/fixing-the-location-of-the-appearance-in-algorithmicx-environment
\usepackage{float}% http://ctan.org/pkg/float

%https://tex.stackexchange.com/questions/25369/how-to-rotate-a-table
\usepackage[graphicx]{realboxes}
\title{7 Quicksort}
\author{Ling Tan}
\date{2018-9-21}

\begin{document}
\maketitle

\section*{7.1 Description of quicksort}
\begin{description}
\item[Divide] Partition (rearrange) the array $A[p..r]$ into two (possibly empty) sub-arrays $A[p..q-1]$ and $A[q+1..r]$ such that each element of $A[p..q-1]$ is less than or equal to $A[q]$, which is, in turn, less than or equal to each element of $A[q+1..r]$. Compute the index q as part of this partitioning procedure.
\item[Conquer] Sort the two sub-arrays $A[p..q-1]$ and $A[q+1..r]$ by recursive calls to quicksort.
\item[Combine] Because the sub-arrays are already sorted, no work is needed to combine them: the entire array $A[p..r]$ is now sorted.
\end{description}
\section*{7.2 Performance of quicksort}
\begin{itemize}
    \item Worst-case partitioning: $T(n)=\Theta(n^2)$
    \item Best-case partitioning: $T(n)=\Theta(n\log n)$
    \item Balanced partitioning: $T(n)=\Theta(n\log n)$
\end{itemize}
\includegraphics[scale=0.7]{"Balanced partitioning"}
\section*{7.3 A randomized version of quicksort}
\begin{algorithm}[H]% Use "stay right HERE" already!
    \caption{RANDOMIZED-PARTITION($A,p,r$)}
    \begin{algorithmic}[1] % The number tells where the line numbering should start
        \State $i=$ RANDOM($p,r$)
        \State exchange $A[r]$ with $A[i]$
        \State \textbf{return }PARTITION($A,p,r$)
    \end{algorithmic}
\end{algorithm}
\section*{7.4 Analysis of quicksort}
\end{document}