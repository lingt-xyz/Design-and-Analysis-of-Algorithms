\documentclass[letter]{book}

\usepackage{fullpage} % Package to use full page
\usepackage{parskip} % Package to tweak paragraph skipping
\usepackage{tikz} % Package for drawing

\usepackage{amssymb}
\usepackage{amsmath}
\usepackage{amscd}

%https://en.wikibooks.org/wiki/LaTeX/Theorems
\usepackage{amsthm}
\theoremstyle{definition}
\newtheorem{theorem}{Theorem}[chapter]
\newtheorem{lemma}[theorem]{Lemma}
\newtheorem{corollary}[theorem]{Corollary}

\theoremstyle{definition}
\newtheorem{mydef}{Definition}[chapter]
\newtheorem*{ex}{Example}

\newtheorem{prop}{Proposition}[chapter]
\theoremstyle{remark}
\newtheorem*{rem}{Remark}

\usepackage{enumitem}

\usepackage{tkz-graph}
\usepackage{float}

\usepackage{hyperref}

\usepackage{MnSymbol}

%https://tex.stackexchange.com/questions/229355/algorithm-algorithmic-algorithmicx-algorithm2e-algpseudocode-confused
\usepackage{algorithm}
\usepackage{algorithmicx}
\usepackage{algpseudocode}

\usepackage{graphicx}
\graphicspath{ {./resources/} }

%https://tex.stackexchange.com/questions/165021/fixing-the-location-of-the-appearance-in-algorithmicx-environment
\usepackage{float}% http://ctan.org/pkg/float

%https://tex.stackexchange.com/questions/25369/how-to-rotate-a-table
\usepackage[graphicx]{realboxes}

\begin{document}
\setcounter{chapter}{25}

\chapter{Maximum Flow}

\section{Flow networks}


\section{The Ford-Fulkerson method}
\begin{lemma}

\end{lemma}
\bigskip
\begin{lemma}

\end{lemma}
\bigskip
\begin{corollary}

\end{corollary}
\bigskip
\begin{lemma}

\end{lemma}
\bigskip
\begin{corollary}

\end{corollary}
\bigskip
\begin{theorem}

\end{theorem}
\bigskip
\begin{lemma}

\end{lemma}
\bigskip
\begin{theorem}

\end{theorem}


\section{Maximum bipartite matching}
\begin{lemma}

\end{lemma}
\bigskip
\begin{theorem}
\end{theorem}
\bigskip
\begin{corollary}

\end{corollary}

\end{document}