\documentclass[a4paper]{article}

\usepackage{fullpage} % Package to use full page
\usepackage{parskip} % Package to tweak paragraph skipping
\usepackage{tikz} % Package for drawing
\usepackage{amsmath}
\usepackage{hyperref}
\usepackage{amssymb}
\usepackage{tikz-qtree}

%https://tex.stackexchange.com/questions/229355/algorithm-algorithmic-algorithmicx-algorithm2e-algpseudocode-confused
\usepackage{algorithm}
\usepackage{algorithmicx}
\usepackage{algpseudocode}

\usepackage{graphicx}
\graphicspath{ {./resources/} }

%https://tex.stackexchange.com/questions/165021/fixing-the-location-of-the-appearance-in-algorithmicx-environment
\usepackage{float}% http://ctan.org/pkg/float

%https://tex.stackexchange.com/questions/25369/how-to-rotate-a-table
\usepackage[graphicx]{realboxes}
\title{17 Amortized Analysis}
\author{Ling Tan}
\date{2018-10-11}

\begin{document}
\maketitle
An amortized analysis \textit{guarantees} the average performance of each option in the worst case. It is different from average case analysis, the probability is not involved.

\subsection*{Two examples used:}
\begin{enumerate}
    \item Stack with MULTIPOP, which can pop several objects at once.
    \item Binary Counter that counts up from 0 by means of operation INCREMENT.
\end{enumerate}
\section*{17.1 Aggregate analysis}
We show that for all $n$, a sequence of $n$ operations takes worst case time $T(n)$ in total. Thus, cost per operation is $T(n)/n$.

\subsection*{Stack operations}
\subsection*{Incrementing a binary counter}

\section*{17.2 The accounting method}
\subsection*{Stack operations}
Assume that:\\
\hspace*{1em} MULTIPOP($S,k$)\\
\hspace*{1em}    for$(i=1; (\text{length}(S)>0)\text{ \&\& }(i\leq k);i++)$\\
\hspace*{2em}    pop(S);

\subsection*{Incrementing a binary counter}

\section*{17.3 The potential method}
\subsection*{Stack operations}
\subsection*{Incrementing a binary counter}
\end{document}