\documentclass[a4paper]{article}

\usepackage{fullpage} % Package to use full page
\usepackage{parskip} % Package to tweak paragraph skipping
\usepackage{tikz} % Package for drawing
\usepackage{amsmath}
\usepackage{hyperref}
\usepackage{amssymb}
\usepackage{tikz-qtree}

%https://tex.stackexchange.com/questions/229355/algorithm-algorithmic-algorithmicx-algorithm2e-algpseudocode-confused
\usepackage{algorithm}
\usepackage{algorithmicx}
\usepackage{algpseudocode}

\usepackage{graphicx}
\graphicspath{ {./resources/} }

%https://tex.stackexchange.com/questions/165021/fixing-the-location-of-the-appearance-in-algorithmicx-environment
\usepackage{float}% http://ctan.org/pkg/float

%https://tex.stackexchange.com/questions/25369/how-to-rotate-a-table
\usepackage[graphicx]{realboxes}
\title{9 Medians and Order Statistics}
\author{Ling Tan}
\date{2018-9}

\begin{document}
\maketitle
\begin{description}
\item[Order Statistics:] The $i$th \textbf{order statistics} of a set of $n$ elements is the $i$th smallest element.
\item[Median:] A median is the “halfway point” of a set.
\item[Selection Problem:] Given an array $A$ of $n$ distinct elements, find $x\in A$ larger than exactly $i-1$ elements in it $\equiv$ find $i$th order statistics of $A$.
\end{description}

\section*{9.1 Minimum and maximum $\Theta(n)$}
\begin{enumerate}
    \item Minimum: with $n-1$ comparisons.
    \item Maximum: same with minimum, with $n-1$ comparison.
    \item Simultaneous minimum and maximum: with $2(n-1)$ comparisons, but can be better.\\
    Process elements in pairs. Get minimum from the smaller group and maximum from the larger group.
\end{enumerate}
\section*{9.2 Selection in \textcolor{red}{expected} linear time $\Theta(n)$}
\begin{itemize}
    \item Quicksort: processes both sides of the partions.
    \item RANDOMIZED-SELECT: works on only one side of the partition.
\end{itemize}
\begin{algorithm}[H]% Use "stay right HERE" already!
    \caption{\textcolor{blue}{RANDOMIZED-SELECT}($A,p,r,i$)}
    \begin{algorithmic}[1] % The number tells where the line numbering should start
        \If{$p==r$} // if only one element
            \State \textbf{return }$A[p]$
        \EndIf
        \State $q=$ RANDOMIZED-PARTITION($A,p,r$)
        \State $k=q-p+1$
        \If{$i==k$} // the pivot value is the answer
            \State \textbf{return } $A[q]$
        \ElsIf{$i<k$}
            \State \textbf{return } RANDOMIZED-SELECT($A,p,q-i,i$)
        \Else
            \State \textbf{return } RANDOMIZED-SELECT($A,q+1,r,i-k)$
        \EndIf
    \end{algorithmic}
\end{algorithm}
\section*{9.3 Selection in worst-case linear time $O(n)$}
\textcolor{blue}{\textit{\textbf{Guarantee}}} a good split upon partitioning the array.\\
\\
\textbf{Algorithm} \textcolor{blue}{SELECT}
\begin{enumerate}
    \item Divide $n$ elements into $\lfloor n/5 \rfloor$ group of $5$ elements, and one group that can have less than $5$ elements.
    \item Find the median of each group.
    \item Use SELECT recursively to find the median $x$ of the $\lceil n/5 \rceil$ values.
    \item Partition the input around $x$.
    \item Continue as in the SELECT algorithm recursively.
\end{enumerate}
\includegraphics[scale=0.5]{"algorithm SELECT"}
\begin{itemize}
    \item At least half of the $\lceil n/5\rceil$ groups contribute at least $3$ elements that are greater than $x$;
    \item Except for the one group that has fewer than $5$ elements,
    \item and the one group containing $x$ itself.
\end{itemize}
\begin{equation*}
    \Rightarrow 3\Bigg(\bigg\lceil\frac{1}{2}\Big\lceil\frac{n}{5}\Big\rceil\bigg\rceil-2\Bigg)\geq \frac{3n}{10}-6
\end{equation*}
\begin{itemize}
    \item At least $3n/10-6$ elements greater than $x$;
    \item At least $3n/10-6$ elements less than $x$;
    \item \textcolor{red}{At most} $n-(3n/10-6)=7n/10+6$ elements less/greater than $x$.
\end{itemize}
\subsection*{Worst case}
\begin{align*}
    T(n)\quad&\leq \quad T(\lceil n/5\rceil)+T(7n/10+6)+O(n)\\
    &\leq \quad c\lceil n/5\rceil+c(7n/10+6)+an\\
    &\leq \quad c(n/5+1)+c(7n/10+6)+an\\
    &\leq \quad cn/5+c+7cn/10+6c+an\\
    &=\quad 9cn/10+7c+an\\
    &=\quad cn+(-cn/10+7c+an)\\
    &=\quad O(n)\quad \text{ if }-cn/10+7c+an\leq 0
\end{align*}
\begin{align*}
    -cn/10+7c+an&\leq 0\\
    -cn+70c+10an&\leq 0\\
    (70-n)c+10an&\leq 0\\
    c&\geq \frac{10an}{(n-70)}\\
    c&\geq 10a\Big(\frac{n}{n-70}\Big)
\end{align*}
\subsection*{9.X Finding a closest pair of points}
\end{document}