\documentclass[a4paper]{article}

\usepackage{fullpage} % Package to use full page
\usepackage{parskip} % Package to tweak paragraph skipping
\usepackage{tikz} % Package for drawing
\usepackage{amsmath}
\usepackage{hyperref}
\usepackage{amssymb}
\usepackage{tikz-qtree}

\title{COMP 465 Assignment 1}
\author{Ling Tan}
\date{2018-09-17}

\begin{document}

\maketitle

It is called \textit{greedy} because when a choice is to be made, it chooses what looks best at the moment.
\section{Problem 1 $ T(n) = 2T(\lfloor{n/2}\rfloor⌋) + n, T(2)=3, T(3)=9$} 

\subsection{$T(n)= \Omega(n\cdot \log n)$}

\textbf{Base:} $T(2)=3, T(3)=9$\\
\textbf{Inductive Hypothesis:} Assume $T(k)\geq c\cdot k \cdot \log{k}, k< n\text{, that is, }T(\lfloor{n/2}\rfloor)\geq c\cdot \lfloor{n/2}\rfloor\log{\lfloor{n/2}\rfloor}$\\
\textbf{Inductive Step:} Must prove $T(n)\geq c\cdot n\log{n}$\\
\textbf{Prove:}
\begin{align*}
    T(n) & = 2T(\lfloor{n/2}\rfloor⌋) + n \\
    & \geq 2(c\lfloor{n/2}\rfloor\log{\lfloor{n/2}\rfloor})+n \\ 
    & \geq 2\Bigl(c \cdot\frac{n-1}{2}\cdot \log{\frac{n-1}{2}}\Bigr)+n \\ 
    & = c \cdot(n-1)\cdot \log{\frac{n-1}{2}}+n \\ 
    & = c \cdot(n-1)\cdot \log{\Bigl(\frac{n-1}{2}\cdot \frac{n}{n}\Bigr)}+n \\
    & = c \cdot(n-1)\cdot \Bigl(\log n - 1 - \log{\frac{n}{n-1}}\Bigr)+n \\ 
    & = cn\log n - cn -cn \log{\frac{n}{n-1}} - c\log n + c + c\log{\frac{n}{n-1}} + n\\
    & = cn\log n - cn \Bigl(1+\log{\frac{n}{n-1}} \Bigr)- c\log n + c + c\log{\frac{n}{n-1}} + n\\
    & \geq cn\log n - cn (1+1)- c\log n + c + c\log{\frac{n}{n-1}} + n& \hfill \log{\frac{n}{n-1}} < 1\\
    & = cn\log n - 2cn - c\log n + c + c\log{\frac{n}{n-1}} + n \\
    & = cn\log n - 2cn - c\log n + c\log{\frac{n}{n-1}}  + c + n \\
    & = cn\log n - 2cn - c(\log n -\log{\frac{n}{n-1}}) + c + n\\
    & = cn\log n - 2cn - c\log ({n-1}) + c + n\\
    & \geq cn\log n - 2cn - cn + c + n & \hfill \log ({n-1})<n\\
    & \geq cn\log n + ( n - 3cn) & \hfill c>0\\
    & = cn\log n + n( 1 - 3c)\\
    & \geq cn \log n \text{ for } c\in (0,1/3]
\end{align*}
\textbf{Boundary:} \\
$T(2)=3\geq c\cdot 2\log 2 \text{ for } c\in (0,1/3]$\\ 
$T(3)=9 \geq c\cdot 3\log3 \text{ for } c\in (0,1/3]$\\
\textbf{Conclusion: }$T(n)= \Omega(n\cdot \log n)\text{ for } c\in (0,1/3] $

\subsection{$T(n)= O(n\cdot \log n)$}

\textbf{Base:} $T(2)=3, T(3)=9$\\
\textbf{Inductive Hypothesis:} Assume $T(k)\leq c\cdot k \cdot \log{k}, k< n\text{, that is, }T(\lfloor{n/2}\rfloor)\leq c\cdot \lfloor{n/2}\rfloor\log{\lfloor{n/2}\rfloor}$\\
\textbf{Inductive Step:} Must prove $T(n)\leq c\cdot n\log{n}$\\
\textbf{Prove:}
\begin{align*}
    T(n) & = 2T(\lfloor{n/2}\rfloor⌋) + n \\
    & \leq 2(c\lfloor{n/2}\rfloor\log{\lfloor{n/2}\rfloor})+n \\ 
    & \leq cn\log{(n/2)}+n \\
    & = cn \log{n} -cn\log{2}+n\\
    & = cn \log n - cn+n \\
    & = cn \log n +(1-c)n \\
    & \leq cn \log n \text{ for } c\in[1,+\infty)
\end{align*}
\textbf{Boundary:} Pick $c=100$\\
$T(2)=3\leq c\cdot 2\log 2 \text{ for } c\in[100,+\infty)$\\ 
$T(3)=9 \leq c\cdot 3\log3 \text{ for } c\in[100,+\infty)$\\
\textbf{Conclusion: }$T(n)= O(n\cdot \log n)\text{ for } c\in[100,+\infty) $

\subsection{$T(n)= \Theta(n\cdot \log n)$}
From $1.1 \text{ and } 1.2$ we can conclude {$T(n)= \Theta(n\cdot \log n)$}

\section{Problem 2 $T(n) = T(n/3) + T(2n/3) + n$} 
Guess $T(n)=\Theta(n\cdot \log n)$

\subsection{$T(n)= \Omega(n\cdot \log n)$}

\textbf{Inductive Hypothesis:} Assume $T(k)\geq c\cdot k \cdot \log{k}, k< n$\\
\textbf{Inductive Step:} Must prove $T(n)\geq c\cdot n\log{n}$\\
\textbf{Prove:}
\begin{align*}
    T(n) & = T(n/3) + T(2n/3) + n \\
    & \geq c\cdot n/3 \cdot \log{n/3} + c\cdot 2n/3 \cdot \log{2n/3} +n \\ 
    & = cn/3 \cdot \log{n} - cn/3 \cdot \log{3} + 2cn/3 \cdot \log{2n} - 2cn/3 \cdot \log{3} +n \\
    & = cn \log n - cn\log 3 + 2cn/3 + n \\
    & = cn \log n +\bigl(n-cn(\log 3 + 2/3)\bigr) \\
    & = cn \log n +n\bigl(1-c(\log 3 + 2/3)\bigr) \\
    & \geq cn \log n \text{ for } c\in (0,\frac{1}{\log 3 + 2/3}]
\end{align*}
\textbf{Conclusion: }$T(n)= \Omega(n\cdot \log n)\text{ for } c\in (0,\frac{1}{\log 3 + 2/3}] $

\subsection{$T(n)= O(n\cdot \log n)$}

\textbf{Inductive Hypothesis:} Assume $T(k)\leq c\cdot k \cdot \log{k}, k< n$\\
\textbf{Inductive Step:} Must prove $T(n)\leq c\cdot n\log{n}$\\
\textbf{Prove:}
\begin{align*}
    T(n) & = T(n/3) + T(2n/3) + n \\
    & \leq c\cdot n/3 \cdot \log{n/3} + c\cdot 2n/3 \cdot \log{2n/3} +n \\ 
    & = cn/3 \cdot \log{n} - cn/3 \cdot \log{3} + 2cn/3 \cdot \log{2n} - 2cn/3 \cdot \log{3} +n \\
    & = cn \log n - cn\log 3 + 2cn/3 + n \\
    & = cn \log n +\bigl(n-cn(\log 3 + 2/3)\bigr) \\
    & = cn \log n +n\bigl(1-c(\log 3 + 2/3)\bigr) \\
    & \leq cn \log n \text{ for } c\in [\frac{1}{\log 3 + 2/3}, +\infty)
\end{align*}
\textbf{Conclusion: }$T(n)= \Omega(n\cdot \log n)\text{ for } c\in [\frac{1}{\log 3 + 2/3}, +\infty) $

\subsection{$T(n)= \Theta(n\cdot \log n)$}
From $2.1 \text{ and } 2.2$ we can conclude {$T(n)= \Theta(n\cdot \log n)$}

\section{Problem 3: 4.3-2} 
The recurrence $T(n)=7T(n/2)+n^2$ describes the running time of an algorithm $A$. A competing algorithm $A'$ has a running time of $T'(n)=aT'(n/4)+n^2$. What is the largest integer value for $a$ such that $A'$ is asymptotically faster than $A$?
\begin{align*}
    & T(n) = 7T (n/2)+n^2 \\
   \Rightarrow &  a=7, b=2, f(n)=n^2 \\
   \Rightarrow &  f(n)/n^{\log_b{a}}=n^{2-\log{7}}\\
   \Rightarrow & \text{case 1} \\
   \Rightarrow & T(n)=\Theta(n^{\log_b{a}})= \Theta(n^{\log{7}})\\\\
    & T'(n) = aT' (n/4)+n^2 \\
   \Rightarrow &  b=4, f(n)=n^2 \\
   \Rightarrow &  f(n)/n^{\log_b{a}}=n^{2-\log_4{a}}
\end{align*}
\begin{equation*}
T'(n) =
\begin{cases}
\Theta(n^2),  & \text{if $a<16$, case 3 } \\
\Theta(n^{\log_b{a}}), & \text{if $a\geq16$, case 1}
\end{cases}
\end{equation*}



When $a<16, \Theta(n^2)<\Theta(n^{\log{7}})$, so $a\geq16$; that is, must find $a\geq16$ such that $\Theta(n^{\log_4{a}}) < \Theta(n^{\log{7}}) $\\
$
\Rightarrow \log_4{a}<\log_2{7}=\log_4{49}\Rightarrow a_{max}=48
$

\section{Problem 4: 7.1-3} 
 Give a brief argument that the running time of PARTITION on a subarray of size n is $\Theta(n)$.\\
Only iterate the array once and only compare once.

\section{Problem 5: 7.2-2} 
What is the running time of QUICKSORT when all elements of array $A$ have the same value?\\
Whatever pivot is chosen, the rest elements would be moved to the same side as they have the same value, so it's the worst case $\Theta(n^2)$.

\section{Problem 6: 9.1-1}
Show that the second smallest of n elements can be found with $n+\lceil{\log{n}}\rceil-2 $ comparisons in the worst case
1. Iterate the set to find the smallest one, this needs $(n-1)$ comparison.\\
2. Find the smallest in the rest $(n-1)$ elements: Divide them into pairs, compare each pair to find the smaller one, then divide the smaller ones to pairs again, and compare each pair to find the smaller one. Recursively do this and finally only two elements are left, then we can find the smallest one in the $(n-1)$ elements in $(\log {\lceil{n-1}\rceil} - 1)$\\
3. In total, used $(n-1) + (\log {\lceil{n-1}\rceil} - 1)=n+\log {\lceil{n-1}\rceil}-2$ comparisons.

\section{Problem 7: 9.3-1} 
In the algorithm SELECT, the input elements are divided into groups of 5. Will the algorithm work in linear time if they are divided into groups of 7? Argue that SELECT does not run in linear time if groups of 3 are used.
\subsection{Group of $7$}
    The number of elements greater (less) than $x$ is at least $4(\bigl\lceil \frac{1}{2}\lceil{\frac{n}{7}}\rceil⌋\bigr\rceil-2⌋)\geq \frac{2}{7}n-8$\\
    So the worst case is $n-(\frac{2}{7}n-8)=\frac{5}{7}n+8$
    \begin{align*}
        T(n) & = T(\lceil{n/7}\rceil⌋)+T(5n/7+8) + O(n) \\
        & \leq cn/7+5cn/7+8c + an \\
        & \leq 6cn/7+8c + an \\
        & = cn + (-cn/7+8c+an)\\
        & \leq cn \text{ if } (-cn/7+8c+an)\leq0 \text{; that is, } c\geq 7a\bigl(n/(n-56)\bigr)
    \end{align*}
    When $n>56$, we can always find a suitable $c$, so it's still linear.
    
\subsection{Group of $3$}
    The number of elements greater (less) than $x$ is at least $2(\bigl\lceil \frac{1}{2}\lceil{\frac{n}{3}}\rceil⌋\bigr\rceil-2⌋)\geq \frac{1}{3}n-4$\\
    So the worst case is $n-(\frac{1}{3}n-4)=\frac{2}{3}n+4\Rightarrow T(n) = T(\lceil{n/3}\rceil⌋)+T(2n/3+4) + O(n)$\\
    Guess $T(n)=\Omega(n\log n)$\\
    Assume $T(k)=\Omega(k\log k), \text{ for } k<n$, so we need to prove
    \begin{align*}
        T(n) & = T(\lceil{n/3}\rceil⌋)+T(2n/3+4) + O(n) \\
        & \geq c(n/3)\log (n/3)+c(2n/3)\log (2n/3) + an \\
        & =\frac{1}{3}cn(\log n -\log 3)+\frac{2}{3}cn(\log 2n -\log 3) + an\\
        & =\frac{1}{3}cn\log n -\frac{1}{3}cn\log 3+\frac{2}{3}cn(\log 2 + \log n -\log 3) + an\\
        & = cn\log n -cn\log 3+\frac{2}{3}cn + an\\
        & \geq cn\log n \text{ if } (-cn\log 3+\frac{2}{3}cn + an)\geq0 \text{; that is, } c\leq \frac{a}{\log 3 - 2/3}
    \end{align*}
    So we can conclude it's not linear.

\subsection{The group size is at least 4}
    \begin{itemize}
        \item At least $\Big\lceil{\frac{k}{2}}\Big\rceil\bigg(\Big\lceil{\frac{1}{2}\big\lceil{\frac{n}{k}}\big\rceil}\Big\rceil-2\bigg)\geq \frac{n}{4}-k$
        \item At most $n-\big(\frac{n}{4}-k\big)=\frac{3n}{4}+k$
        \item What the value of $k$ to get $O(n)$?
    \end{itemize}
    \begin{align*}
        T(n)&= T\Big(\Big\lceil{\frac{n}{k}}\Big\rceil\Big)+T\Big(\frac{3n}{4}+k\Big)+O(n)\\
        &\leq c\Big(\frac{n}{k}+1\Big)+c\Big(\frac{3n}{4}+k\Big)+O(n)\\
        &=\frac{cn}{k}+c+\frac{3cn}{4}+ck+O(n)\\
        &=cn\Big(\frac{1}{k}+\frac{3}{4}\Big)+c+ck+O(n)
    \end{align*}

\section{Problem 8: 9.3-8} 
1. Prove the median of $2n$ is between the median of $A$ and the median of $B$.\\
Assume the median of $2n$ is $A_k$ at the $k^{th}$ of $A$.\\
$\Rightarrow A \text{ contains } k-1 \text{ elements less than } A_k$\\
$\Rightarrow B \text{ contains } n-k \text{ elements less than } A_k (\text{because in total we have } n-1 \text{ elements less than } A_k)$\\
$\Rightarrow$ If $k<n/2\Rightarrow A_{median}>A_k$, and $B_{median}<A_k$; If $k>n/2\Rightarrow A_{median}<A_k$, and $B_{median}>A_k$.\\
$\Rightarrow A_k$ is between the $A_{median}$ and $B_{median}$.\\

2. Find the median of $2n$.\\
Based on the previous conclusion, we can find $A_k$ by recursively compare $A_{median}$ and $B_{median}$.\\
\textbf{Algorithm} $Median(A,B,n)$\\
\textbf{Input} Arrays $A, B$ and their size $n$\\
\textbf{Outpu} The median of arrays $A$ and $B$\\
\hspace*{1cm}if $(n==1)$\\
\hspace*{2cm}return $min(A_1, B_1)$\\
\hspace*{1cm}else if $A_{\lfloor{n/2}\rfloor} < B_{\lfloor{n/2}\rfloor}$\\
\hspace*{2cm}return $Median(A_{(\lfloor{n/2}\rfloor+1)\dots n}, B_{1\dots(\lfloor{n/2}\rfloor-1)}, n/2)$\\
\hspace*{1cm}else\\
\hspace*{2cm}return $Median(A_{1\dots(\lfloor{n/2}\rfloor-1)}, B_{(\lfloor{n/2}\rfloor+1)\dots n}, n/2)$\\

\end{document}