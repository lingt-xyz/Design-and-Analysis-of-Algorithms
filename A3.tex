\documentclass[a4paper]{article}

\usepackage{fullpage} % Package to use full page
\usepackage{parskip} % Package to tweak paragraph skipping
\usepackage{tikz} % Package for drawing

\usepackage{hyperref}
\usepackage{amsmath}
\usepackage{amssymb}
\usepackage{amsthm}
\usepackage{enumitem}

\usepackage{subcaption}

\title{COMP 465 Assignment 3}
\author{Ling Tan}
\date{2018/11/05}

\begin{document}

\maketitle

\section{17.1-2} Show that if a DECREMENT operation were included in the $k$-bit counter example, $n$ operations could cost as much as $\Theta(nk)$ time.\\
\textcolor{blue}{Answer:}

\section{17.2-1} A sequence of stack operations is performed on a stack whose size never exceeds $k$. After every $k$ operations, a copy of the entire stack is made for backup purposes. Show that the cost of $n$ stack operations, including copying the stack, is $O(n)$ by assigning suitable amortized costs to the various stack operations.\\
\textcolor{blue}{Answer:}

\section{22.5-7} A directed graph $ G= (V, E)$ is said to be \textit{semiconnected} if, for all pairs of vertices $u, v \in V$, we have path from $u$ to $v$ or from $v$ to $u$. Give an efficient algorithm to determine whether or not $G$ is semiconnected. Prove that your algorithm is correct, and analyze its running time.\\
\textcolor{blue}{Answer:}

\section{} Show that a graph has a unique minimum spanning tree if all the weights of $G$ are distinct.\\
\textcolor{blue}{Answer:}

\section{24.1-1} Run the Bellman-Ford algorithm on the directed graph of Figure 24.4, using vertex $z$ as the source. In each pass, relax edges in the same order as in the figure, and show the $d$ and $\pi$ values after each pass. Now, change the weight of edge $(z, x)$ to $4$ and run the algorithm again, using $s$ as the source.\\
\textcolor{blue}{Answer:}

\section{25.2-6} How can the output of the Floyd-Warshall algorithm be used to detect the presence of a negative-weight cycle?\\
\textcolor{blue}{Answer:}

\section{26.2-2} Show the execution of the Edmonds-Karp algorithm on the flow network of Figure 26.1(a).\\
\textcolor{blue}{Answer:}

\section{26.3-1} Run the Ford-Fulkerson algorithm on the flow network in Figure 26.8(b) and show the residual network after each flow augmentation. Number the vertices in $L$ top to bottom from $1$ to $5$ and in $R$ top to bottom from $6$ to $9$. For each iteration, pick the augmenting path that is lexicographically smallest.\\
\textcolor{blue}{Answer:}

\end{document}

