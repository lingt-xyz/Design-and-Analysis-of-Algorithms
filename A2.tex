\documentclass[a4paper]{article}

\usepackage{fullpage} % Package to use full page
\usepackage{parskip} % Package to tweak paragraph skipping
\usepackage{tikz} % Package for drawing
\usepackage{amsmath}
\usepackage{hyperref}
\usepackage{amssymb}
\usepackage{tikz-qtree}
%https://tex.stackexchange.com/questions/165021/fixing-the-location-of-the-appearance-in-algorithmicx-environment
\usepackage{float}% http://ctan.org/pkg/float

\title{COMP 465 Assignment 2}
\author{Ling Tan}

\begin{document}

\maketitle

\section*{Problem 1: (15.2-1)} Find an optimal parenthesization of a matrix-chain product whose sequence of dimensions is $\langle5, 10, 3, 12, 5, 50, 6\rangle$
\subsection*{Answer:}
\begin{table}[H]
    \centering
    $
    \begin{array}{c|cccccc}
     & 1 & 2 & 3 & 4 & 5 & 6\\
    \hline
     1 & 0 & 150 & 3 & 4 & 5 & 6\\
     2 &   & 0 & 360 & 4 & 5 & 6\\
     3 &   &   & 0 & 180 & 5 & 6\\
     4 &   &   &   & 0 & 3000 & 6\\
     5 &   &   &   &   & 0 & 1500\\
     6 &   &   &   &   &   & 0
    \end{array}
    $
    \caption{$m$ table}
    \label{tab:my_label}
\end{table}
\begin{table}[H]
    \centering
    $
    \begin{array}{c|cccccc}
     & 1 & 2 & 3 & 4 & 5 & 6\\
    \hline
     1 & 0 & 1 & 3 & 4 & 5 & 6\\
     2 &   & 0 & 2 & 4 & 5 & 6\\
     3 &   &   & 0 & 3 & 5 & 6\\
     4 &   &   &   & 0 & 4 & 6\\
     5 &   &   &   &   & 0 & 5\\
     6 &   &   &   &   &   & 0
    \end{array}
    $
    \caption{$s$ table}
    \label{tab:my_label}
\end{table}


\section*{Problem 2: (15.4-2) } Show how to reconstruct an LCS from the completed $c$ table and the original sequences $X = \langle x_1, x_2,\dots, x_m\rangle$ and $Y = \langle y_1, y_2, \dots, y_n\rangle$ in $O(m +n)$ time, without using the $b$ table.
\subsection*{Answer:}

\section*{Problem 3: (15.4-5)} Give an $O(n^2)$-time algorithm to find the longest monotonically increasing subsequence of a sequence of $n$ numbers.
\subsection*{Answer:}
Be monotone increasing means sorted. Therefore this problem is equivalent to finding the longest sorted (increasingly) subsequence in this sequence.\\
First, construct a increasingly sorted sequence of the original sequence.\\
Second, apply LCS algorithm on these two list.\\
Because this is $n\times n$ array, we have $O(n^2)$.

\section*{Problem 4: (16.1-2)} Suppose that instead of always selecting the first activity to finish, we instead select the last activity to start that is compatible with all previously selected activities. Describe how this approach is a greedy algorithm, and prove that it yields an optimal solution.

\section*{Problem 5: (16.2-2)} Give a dynamic-programming solution to the $0-1$ knapsack problem that runs in $O(n W)$ time, where $n$ is number of items and $W$ is the maximum weight of items that the thief can put in his knapsack.

\section*{Problem 6: (16.3-1)} Prove that a binary tree that is not full cannot correspond to an optimal prefix code.

\section*{Problem 7: (23.1-5)} Let $e$ be a maximum-weight edge on some cycle of $G = (V, E)$. Prove that there is a minimum spanning tree of $G′ = (V, E -\{e\})$ that is also a minimum spanning tree of $G$. That is, there is a minimum spanning tree of G that does not include $e$.
 
\section*{Problem 8: (23.2-8)} Professor Toole proposes a new divide-and-conquer algorithm for computing minimum spanning trees, which goes as follows. Given a graph $G = (V, E)$, partition the set $V$ of vertices into two sets $V_1$ and $V_2$ such that $|V1|$ and $|V2|$ differ by at most $1$. Let $E_1$ be the set of edges that are incident only on vertices in $V_1$, and let $E_2$ be the set of edges that are incident only on vertices in $V_2$. Recursively solve a minimum-spanning-tree problem on each of the two subgraphs $G1 = (V1, E1)$ and $G2 = (V2, E2)$. Finally, select the minimum-weight edge in $E$ that crosses the cut $(V_1, V_2)$, and use this edge to unite the resulting two minimum spanning
trees into a single spanning tree.

\end{document}